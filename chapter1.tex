\chapter{Pandas Foundations\label{chapter1}}
\section{Importing pandas}

\section{Introduction}

\section{The pandas DataFrame}

\section{DataFrame attributes}

\section{Understanding data types\label{Understanding data types}}

\section{Selecting a column}

\section{Calling Series methods\label{Calling Series methods}}

\section{Series operations}

\section{Chaining Series methods}

\section{Renaming column names}

\section{Creating and deleting columns}
pandas has a few different ways to add new columns to a DataFrame.
In this recipe, we create new columns in the movie dataset by using the \texttt{.assign} method and then delete columns with the \texttt{.drop} method.

\subsection{How to do it...}
\begin{enumerate}
\item


One way to create a new column is to do an index assignment. Note that this will not return a new DataFrame but mutate the existing DataFrame. If you assign the column to a scalar value, it will use that value for every cell in the column. 
\begin{py}{}

\end{py}
\item While this method works and is common, as I find myself chaining methods very often, I prefer to use the .assign method instead. This will return a new DataFrame with the new column. Because it uses the parameter name as the column name, the  column name must be a valid parameter name:
\begin{py}{}
movies = pd.read_csv('movie.csv')
idx_map = {
    'Avatar': 'Ratava',
    'Spectre': 'Ertcepes',
    "Pirates of the Caribbean: At World's End": 'POC',
}
col_map = {
    'aspect_ratio': 'aspect',
    'movie_facebook_likes': 'fblikes',
}
(
    movies.rename(index=idx_map, columns=col_map
                 ).assign(has_seen=0)
)
\end{py}
\item Let's 
add up all actor and director Facebook like columns and assign them to the \texttt{total\_likes} column. We can do this in a couple of ways.
\begin{py}{}
total = (
    movies['actor_1_facebook_likes']
    + movies['actor_2_facebook_likes']
    + movies['actor_3_facebook_likes']
    + movies['director_facebook_likes']
)
total.head(5)
\end{py}
My preference is to use methods that we can chain, so I prefer calling .sum here. 
I will pass in a list of columns to select to .loc to pull out just those columns that 
I want to sum:
\begin{py}{}
cols = [
    'actor_1_facebook_likes',
    'actor_2_facebook_likes',
    'actor_3_facebook_likes',
    'director_facebook_likes'
]
sum_cols = movies.loc[:, cols].sum(axis='columns')
sum_cols.head()
\end{py}
Then we can assign this Series to the new column. Note that when we called the 
+ operator, the result had missing numbers (NaN), but the .sum method ignores 
missing numbers by default, so we get a different result:
\begin{py}{}
movies.assign(total_likes=sum_cols).head()
\end{py}

Another option is to pass in a function as the value of the parameter in the call to the \texttt{.assign} method. This function accepts a DataFrame as input and should 
return a Series:
\begin{py}{}
def sum_likes(df):
    return df[
        [
            c
            for c in df.columns 
            if 'like' in c and ('actor' in c or 'director' in c)
        ]
    ].sum(axis=1)
movies.assign(total_likes=sum_likes).head(3)
\end{py}
\item
When numeric columns are added to one another as in the 
preceding step using the plus operator, the result is \texttt{NaN} if there is any value missing. 
However, with the \texttt{.sum} method it converts \texttt{NaN} to zero.
\begin{py}{}
(
    movies.assign(total_likes=sum_likes)['total_likes']
    .isna()
    .sum()
)

(
    movies.assign(total_likes=total)['total_likes']
    .isna()
    .sum()
)
\end{py}
We could fill in the missing values with zero as well:
\begin{py}{}
(
    movies.assign(total_likes=total.fillna(0))[
        'total_likes'
    ]
    .isna()
    .sum()
)
\end{py}


\end{enumerate}


\subsection{There's more...}

It is possible to insert a new column into a specific location in a DataFrame with the .insert
method. The .insert method takes the integer position of the new column as its first 
argument, the name of the new column as its second, and the values as its third. You will 
need to use the \texttt{.get\_loc} Index method to find the integer location of the column name.
The .insert method modifies the calling DataFrame in-place, so there won't be an 
assignment statement. It also returns None.
\begin{py}{}
movies.insert(
    loc=profit_index,
    column='profit',
    value=movies['gross'] - movies['budget'],
)
\end{py}


An alternative to deleting columns with the .drop method is to use the del statement. This 
also does not return a new DataFrame, so favor .drop over this:
\begin{py}{}
del movies['director_name']
\end{py}